%%%%
%%%% Multiagent Simulation and the MASON Library
%%%% GeoMason GIS extension
%%%%
%%%% Copyright 2012 by Mark Coletti
%%%%
%%%% LaTeX Source
%%%% This source code, and embedded PDFs and sources (such as OmniGraffle Files)
%%%% Are distributed under the Academic Free License version 3.0
%%%% See the file "LICENSE" for more information
%%%%
%%%% When you build this source code, the resulting PDF file is licensed under the
%%%% Creative Commons Attribution-No Derivative Works 3.0 United States License
%%%% See the URL http://creativecommons.org/licenses/by-nd/3.0/us/   for more information
%%%%
%%%% If you have any questions, feel free to contact me at mcoletti@gmail.com

\documentclass[twoside,10pt]{book}
\usepackage{fullpage}
\usepackage{mathpazo}
\usepackage[noend]{algpseudocode}
\usepackage{amsmath}
\usepackage{latexsym}
\usepackage{graphicx}
\usepackage{wrapfig}
\usepackage{bm}
\usepackage{qtree}
\usepackage{array}
\usepackage{eurosym}
\usepackage{textcomp}
\usepackage{makeidx}
\usepackage{rotating}
\usepackage{multirow}
\usepackage{multicol}
\usepackage{microtype}
\usepackage{afterpage}
\usepackage{color}\definecolor{gray}{gray}{0.5}
\usepackage{xcolor}
\usepackage{alltt}
\usepackage{fancyvrb}
\usepackage[font=footnotesize,labelsep=quad,labelfont=it]{caption}
%%% Added in order to use hyperref -- this stuff has to appear before bibentry,x
%%% which has a conflict with regard to \bibitem.  See later in this file for more stuff that has
%%% to be added afterwards
  \makeatletter
  \let\saved@bibitem\@bibitem
  \makeatother

\usepackage{bibentry}
\usepackage[hyperfootnotes=false,linktocpage=true,linkbordercolor={0.5 0 0}]{hyperref}
%%% Note that to avoid a link being created from \pageref, just use \pageref*
%%% End hyperref stuff

\renewcommand\textfraction{0.0}
\renewcommand\topfraction{1.0}
\renewcommand\bottomfraction{1.0}

\makeindex

\newcommand\file[1]{\textsf{#1}}
\newcommand\variable[1]{\textsf{#1}}
%\newcommand\package[1]{\textsf{#1}}
\newcommand\package[1]{\index{Packages!{#1}}\textsf{#1}}
\newcommand\Package[1]{\index{Packages!{#1}|textbf}\textsf{#1}}
%\newcommand\class[1]{\textsf{#1}}
\newcommand\class[1]{\index{Classes!{#1}}\textsf{#1}}
\newcommand\Class[1]{\index{Classes!{#1}|textbf}\textsf{#1}}
\newcommand\method[1]{\textsf{#1}}
\newcommand\parameter[1]{\texttt{#1}}
\newcommand\character[1]{\texttt{"{#1}"}}
\newcommand\textstr[1]{\texttt{"{#1}"}}
\newcommand\code[1]{\textsf{#1}}
%\newcommand\code[1]{\texttt{#1}}

\newcommand\ignore[1]{}


\newcommand\sidebara[3]{\begin{wrapfigure}{r}[0in]{3.2in}%
\vspace{-1.1em}\hfill\framebox{\begin{minipage}{3in}\setlength\parindent{1.5em}\footnotesize{\noindent\textit{#1}

\vspace{0.5em}{\noindent #2}}
\end{minipage}}
\vspace{#3}
\end{wrapfigure}
}

\newcommand\sidebar[2]{\begin{wrapfigure}{r}[0in]{3.2in}%
\vspace{-1.1em}\hfill\framebox{\begin{minipage}{3in}\setlength\parindent{1.5em}\footnotesize{\noindent\textit{#1}

\vspace{0.5em}{\noindent #2}}
\end{minipage}}
\vspace{-0.5em}
\end{wrapfigure}
}



%%% Hack to allow more spacing before and after an hline
\newcommand\tstrut{\rule{0pt}{2.4ex}}
\newcommand\bstrut{\rule[-1.0ex]{0pt}{0pt}}

% Increase the numbering depth
\setcounter{secnumdepth}{3}
\setcounter{tocdepth}{6}


%%%% This code is used to create consistent lists of methods

% From TUGboat, Volume 24 (2003), No. 2 "Hints & Tricks"
\newcommand*{\xfill}[1][0pt]{%
	\cleaders
		\hbox to 1pt{\hss
			\raisebox{#1}{\rule{1.2pt}{0.4pt}}%
			\hss}\hfill}
			
\newenvironment{methods}[1]{
\vspace{1.0em}\noindent\textsf{\textbf{#1 Methods}}\quad \xfill[0.5ex]
\vspace{-0.25em}
\begin{description}
\small}
{\end{description}\hrule\vspace{1.5em}}

\newcommand{\mthd}[1]{\item[{\sf #1}]~\newline}


\newcommand\reference[1]{\vspace{0.5em}\hfill{\parbox{6in}{\raggedleft\noindent\textsf{#1}}}}

% Include subsubsection in the TOC
\setcounter{tocdepth}{3}

% Use with a %, like this:   \params{%
\newcommand\params[1]{\vbox{\begin{quote}\small\tt{\noindent{#1}}\end{quote}}}
\newcommand\script[1]{\params{#1}}
\newcommand\java[1]{\params{#1}}

% Allow poor horizontal spacing
\sloppy

% Allow a ragged bottom even in two-sided
\raggedbottom

% Command to push text to following page without the cutoff that occurs with clearpage
\newcommand\bump{\vspace{10in}}

% Command to push text to following line
\newcommand\hbump{\hspace{10in}}


% Define an existing word in text as an index item
\newcommand{\idx}[1]{\index{#1}#1}

% Define an existing word in text as an index item and make it bold
\newcommand{\df}[1]{\index{#1}\textbf{#1}}

% Provide a separate index item for a word in text and make it bold
\newcommand{\dfa}[2]{\index{#1}\textbf{#2}}

% Create algorithms and definitions
\newtheorem{algm}{Algorithm}
\newtheorem{defn}{Definition}

% Initial figures, pages, algorithms, and sections should be 0 :-)
\setcounter{figure}{-1}	% Mona is Figure 0
\setcounter{page}{-1}	% Start with Page 1 (the Front Page).  I'd like it to be Page 0 but it messes up twosided
\setcounter{algm}{-1}	% Start with Algorithm 0 (the Example Algorithm)
\setcounter{section}{-1}	% Start at Section 0 (the Introduction)

\thispagestyle{plain}
\thispagestyle{empty}

\newcommand\hsp[1]{{\rule{0pt}{0pt}\hspace{#1}}}
\newcommand\spc{{\rule{0pt}{0pt}~}}



\makeindex


\begin{document}

\begin{wrapfigure}{r}[2.5in]{4in}
\vspace{-1.1in}\includegraphics[height=11in]{campus.pdf}
\end{wrapfigure}

\begin{flushleft}
\huge\bf The GeoMason Cookbook\\[\baselineskip]
\end{flushleft}

\noindent\Large\bf Mark Coletti\\
{\large\rm 
Department of Computer Science\\
George Mason University}
\\
\\
\\
\Large\bf Zeroth Edition\\
\large\rm Online Version 0.0\\
\large\rm September, 2012\\

\vspace{5in}
\noindent\Large\bf Where to Obtain GeoMason\\
\large\rm http:/\!/cs.gmu.edu/\!\(\sim\)eclab/projects/mason/extensions/geomason/

\clearpage

\small 
\noindent {\Large\bf Copyright }  2012 by Mark Coletti.

\vspace{0.25in}
\noindent {\Large\bf Thanks to } Sean Luke, Andrew Crooks, Keith Sullivan

\vspace{0.25in}

\noindent {\Large\bf Get the latest version of this document or suggest improvements here:}

\reference{http:/\!/cs.gmu.edu/\!\(\sim\)eclab/projects/mason/extensions/geomason}

\vspace{0.15in}

\vspace{0.15in}
	\noindent {\Large\bf This document is licensed} under the {\bf Creative Commons Attribution-No Derivative Works 3.0 United States License,} except for those portions of the work licensed differently as described in the next section. To view a copy of this license, visit http:/\!/creativecommons.org/licenses/by-nd/3.0/us/ or send a letter to Creative Commons, 171 Second Street, Suite 300, San Francisco, California, 94105, USA.  A quick license summary:
	\begin{itemize}
	\item You are free to redistribute this document.
	\vspace{-0.5em}\item {\bf You may not} modify, transform, translate, or build upon the document except for personal use.   
	\vspace{-0.5em}\item You must maintain the author's attribution with the document at all times.
	\vspace{-0.5em}\item You may not use the attribution to imply that the author endorses you or your document use.  
	\end{itemize}
	This summary is just informational: if there is any conflict in interpretation between the summary and the actual license, the actual license always takes precedence.


\normalsize
\cleardoublepage

\tableofcontents
\clearpage


\chapter{Introduction}
\label{ch:intro}

\begin{figure}[h]\vspace{-33em}\hspace{30em}\includegraphics[width=4in]{campus.pdf}\vspace{2em}\end{figure}

GeoMason is a MASON extension that adds basic geospatial capability.

\section{Architectural Layout}

%\begin{wrapfigure}{r}[0in]{3.7in}
%\vspace{-4em}\hfill\includegraphics[width=3.7in]{MASONLayout.pdf}\vspace{-3em}\end{wrapfigure}

% TODO Greatly expand this introduction.

MASON is a sophisticated multi-agent simulation library.
Unfortunately it does not natively support geospatial data.  GeoMason
is a MASON extension that embues MASON with some limited geospatial
awareness.  With GeoMason one is able to load, display, and manipulate
data that is, in some way, grounded to the Earth's surface.  This
Cookbook provides a set of ``recipes'' for using GeoMason.


\chapter{Reading and Writing Geospatial Data}
\label{ch:io}

This chapter covers recipes for reading and writing vector and grid based geospatial data.

\section{Reading Geospatial Data}
\label{sec:readingdata}
This section covers reading geospatial data into MASON using GeoMason.

\subsection{Reading a Shape File}
\label{sub:readingshapefiles}

\begin{description}
\item[Problem] ~\\
You want to read vector geospatial data stored in a Shape
file.\index{Shape files!reading|(}

\item[Solution]~\\
Create a \class{GeomVectorField} and use \method{ShapeFileImporter.read()} to load data into it.
\begin{Verbatim}[frame=lines,framesep=5mm,numbers=left,commandchars=+\[\]]
GeomVectorField vectorField = new GeomVectorField();

try {
   ShapeFileImporter.read("file:foo.shp", vectorField);
} catch (FileNotFoundException ex)
{  /* handle exception */  }
\end{Verbatim}

\item[Discussion]~\\
Though there exist other GeoMason classes capable of reading Shape
files --- \class{GeoToolsImporter} and \class{OGRImporter} --- the
native GeoMason shape file importer, \class{ShapeFileImporter}, is
recommended, especially given that it has no third party dependencies
as the other importer classes do.

Given the general static nature of shape files, the above code snippet
is likely to be in a \class{SimState} subclass constructor.
Alternatively you may place it in the \method{start()} though be mindful
that means that the shape file will be loaded again each time the
simulation is restarted.

Note that the units of the loaded vector layer will be those of the
underyling coordinate reference system.  So if the shape file is
in meters, such as is found in data in Universal Transverse Mercator (UTM), then all loaded
geometry will similarly be in meters.   Also note that GeoMason uses
the \idx{Java Topology Suite} (JTS) to store all the geometry.  JTS uses a
flat Cartesian plane for all points; so be aware of this when loading
data from a non-planar reference system.  That is, if you na\"{i}vely
load, say, native lat/lon data, which corresponds to coordinates along
a ellipsoid, that you will have introduced distortions in the implicit
projection you have just done to a 2D plane.  Moreover, these
distortions will be more pronounced for large surface areas.
\end{description}



\subsection{Reading Multiple Vector Layers}
\label{sub:multiplevectorlayers}

\begin{description}
\item[Problem]~\\
You want to read in more than one thematic layer of vector
data.

\item[Solution]~\\
After ensuring that each layer uses the same coordinate reference
system, read in each layer, and then synchronize the minimum bounding
rectangles (MBR) for all the layers.\index{minimum bounding rectangle}
\begin{Verbatim}[frame=lines,framesep=5mm,numbers=left,commandchars=+\[\]]
GeomVectorField firstVectorField = new GeomVectorField();
GeomVectorField secondVectorField = new GeomVectorField();
GeomVectorField thirdVectorField = new GeomVectorField();

try {
   ShapeFileImporter.read("file:foo.shp", firstVectorField);
   ShapeFileImporter.read("file:bar.shp", secondVectorField);
   ShapeFileImporter.read("file:baz.shp", thirdVectorField);
} catch (FileNotFoundException ex)
{  /* handle exception */  }

+color[red]Envelope globalMBR = firstVectorField.getMBR();+label[ex:MBRstart]

+color[red]globalMBR.expandToInclude(secondVectorField.getMBR());
+color[red]globalMBR.expandToInclude(thirdVectorField.getMBR());

+color[red]firstVectorField.setMBR(globalMBR);
+color[red]secondVectorField.setMBR(globalMBR);
+color[red]thirdVectorField.setMBR(globalMBR);+label[ex:MBRend]
\end{Verbatim}

\item[Discussion]~\\
It is possible that the disparate shape files may have different
coordinate reference systems, as can happen if the shape files came
from different sources.  It is vitally important to ensure that all
the layers have the same coordinate reference system before being
loaded into GeoMason.  For example, a vector layer that uses lat/lon coordinates
will have radically different geometry values from another vector layer that
uses UTM even though they may cover the same area on Earth.
Essentially, GeoMason is not a GIS so it will not do
on-the-fly projections of the data.  Users can use a real GIS tool,
such as QuantumGIS \footnote{http://www.qgis.org/}, to manually
reproject data prior to loading into GeoMason.

It is important to ensure that all the layers have the same MBR
otherwise they will not align properly when displayed.  Naturally, this
is optional if you do not intend on rendering the layers.  Regardless
it would be prudent to do so anyway on the chance that later you
change your mind and want to see the \class{GeoVectorField}s.  The
highlighted lines \ref{ex:MBRstart}-\ref{ex:MBRend} show how to
synchronize the MBRs between loaded \class{GeomVectorField}s.
Basically, you get the MBR of the first \class{GeomVectorField}, expand
it to include the area of the MBRs for the remaining
\class{GeomVectorField}s, and then set them all to the one
all-inclusive MBR.

As noted in recipe \ref{sub:readingshapefiles}, given the general
static nature of shape files, this code snippet is likely to be done
in the \class{SimState} constructor; however, again, the layers could
also be loaded via \method{start()}, though that means loading the shape
files every time the simulation is re-run.
\end{description}



\subsection{Reading a Shape File and Some of Its Attributes}
\label{sub:readingshapefileattributes}

\begin{description}
\item[Problem]~\\
You want to read a Shape file and only some of its associated
attributes.\index{Shape files!reading|)}\index{Shape files!attributes}

\item[Solution]~\\
Read in a shape file as in recipe \ref{sub:readingshapefiles}, but
specify the desired attributes by creating a \class{Bag} of \code{String}s
containing attribute names, and then passing that \class{Bag} to
\method{ShapeFileImporter.read()}.

\begin{Verbatim}[frame=lines,label=\textsf{Reading Attributes},framesep=5mm,numbers=left,commandchars=+\[\]]
GeomVectorField vectorField = new GeomVectorField();

+color[red]Bag desiredAttributes = new Bag();+label[ex:attributestart]
+color[red]desiredAttributes.add("NAME");
+color[red]desiredAttributes.add("TYPE");+label[ex:attributeend]

try {
   +color[red]ShapeFileImporter.read("file:foo.shp", vectorField, desiredAttributes);+label[ex:attributeread]
} catch (FileNotFoundException ex)
{  /* handle exception */  }
\end{Verbatim}

Each spatial object is wrapped in a \class{MasonGeometry} object
which, in turn, also stores any associated attributes.  Use the
appropriate \class{MasonGeometry} \code{get*Attribute()} method to
retrieve the attribute value.

\begin{Verbatim}[frame=lines,label=\textsf{Using Attributes},framesep=5mm,numbers=left]
Bag geometries = vectorField.getGeometries();

for (int i = 0; i < geometries.size(); i++)
{
    MasonGeometry geometry = (MasonGeometry) geometries.objs[i];

    int type = geometry.getIntegerAttribute("TYPE");
    String name = geometry.getStringAttribute("NAME");
}
\end{Verbatim}

\item[Discussion]~\\
Most shape files have an associated set of attributes describing each
feature.  For example, buildings will have names, roads will have a
number of lanes, bridges will have a type, and so on.  These
attributes can be strings, numbers, or boolean values.

By default GeoMason will not load any associated attributes --- if you
want any attributes you will have to ask for them by name.  You do
this by filling a \code{Bag} with strings of desired attribute names,
and then passing that \code{Bag} to a
\method{ShapeFileImporter.read()} invocation, as shown in the
highlighted lines
\ref{ex:attributestart}-\ref{ex:attributeend} and
\ref{ex:attributeread} in the code example ``Reading Attributes.'' This will then load
each \class{MasonGeometry} object that corresponds to each spatial
entity with a set of attribute/value pairs.  These attributes can be
later retrieved with an appropriate call to
\method{getStringAttribute()}, \method{getIntegerAttribute()}, or
\method{getDoubleAttribute()}; you can also invoke
\method{getAttribute()} to retrieve the value object directly.  An
example of using these methods is shown in the code snippet ``Using Attributes.''

Unfortunately this does mean you have to know ahead of time the
available attributes and their respective names.  If you do not know
the available attributes, you can use a GIS such as QuantumGIS or
ArcGIS to discover the attribute names.
\end{description}






\subsection{Reading an ARC/Info ASCII Grid File}
\label{sub:readinggridfile}

\begin{description}
\item[Problem]~\\
You want to read an Arc/Info ASCII Grid file.

\item[Solution]~\\
Create a \class{GeomGridField} and an \code{InputStream} opened on the grid file, then use \class{ArcInfoASCGridImporter()} to load the data into the grid field from the open input stream.
\begin{Verbatim}[frame=lines,framesep=5mm,numbers=left,commandchars=+\[\]]
	GeomGridField gridField = new GeomGridField();
	
	InputStream inputStream = new FileInputStream("foo.asc");
	ArcInfoASCGridImporter.read(inputStream, GridDataType.INTEGER, gridField);
\end{Verbatim}

\item[Discussion ]~\\
Essentially a \class{GeomGridField} is a wrapper round a MASON
\class{Grid2D} object that embues some limited geospatial
characteristics -- essentially it's a georeferenced  MASON \class{Grid2D}.

\method{ArcInfoASCGridImporter.read()} will use one of two
\class{Grid2D} MASON subclasses, \class{IntGrid2D} or
\class{DoubleGrid2D},  depending
on whether integer or real-value data is used, respectively.  Unfornately GeoMason is not smart enough to
figure out ahead of time which underlying MASON \class{Grid2D}
subclass to use so you have to specify that via the \class{GridDataType}
argument to \method{ArcInfoASCGridImporter.read()}; i.e.,
\code{GridDataType.INTEGER} for integer based grid data or
GridDataType.DOUBLE for real-value based grid data.  

Many times you can readily intuit the underlying data type of an
ASC/Grid file.  E.g., a grid of population values such as Landscan
values \footnote{http://www.ornl.gov/sci/landscan/} will likely use integers,
wheres one of elevation postings, such as found in \idx{Digital Elevation
Models}, will likely use floating point values.  However, if you do not know
whether integers or floats are used in a given grid file, you can peep
at it with a text editor to find out.  If you have a UNIX-like command line, you
can use ``\code{head -7}'' to look at the first seven lines of text in an
ASC/Grid file. Regardless of how you look at the data, it
should be readily apparently whether the file contains all integers or floats.
\end{description}


\subsection{Reading a Mix of Grid and Vector Data}
\label{sub:readingmixofdata}

\begin{description}
\item[Problem]~\\
You want to read in multiple layers that are a mix of vector and grid geospatial data.

\item[Solution]~\\
After ensuring that all the layers have the same coordinate reference system, read all the layers into GeomVectorField or GeomGridFields, as appropriate, and then synchronize their respective minimum bounding rectangles.
\begin{Verbatim}[frame=lines,framesep=5mm,numbers=left,commandchars=+\[\]]
GeomVectorField vectorField = new GeomVectorField();
GeomGridField gridField = new GeomGridField();

try {
   ShapeFileImporter.read("file:vector.shp", firstVectorField);

   InputStream inputStream = new FileInputStream("grid.asc");
   ArcInfoASCGridImporter.read(inputStream, GridDataType.INTEGER, gridField);
} catch (FileNotFoundException ex)
{  /* handle exception */  }

Envelope globalMBR = vectorField.getMBR();

globalMBR.expandToInclude(gridField.getMBR());

vectorField.setMBR(globalMBR);
gridField.setMBR(globalMBR);
\end{Verbatim}

\item[Discussion ]~\\
The same issues apply here as noted in recipe
\ref{sub:multiplevectorlayers} --- i.e., not only do the coordinate
reference systems need to be identical between layers, but the MBRs
also need to be synchronized.  Also, as in recipe
\ref{sub:readinggridfile}, you will have to specify the appropriate
\class{GridDataType} that corresponds to type of data found in the
grid file.

One typical scenario this recipe covers is overlaying political
boundaries over grid data.
\end{description}



% TODO finish this recipe
\subsection{Reading Other Kinds of Geospatial Data}
\label{sub:readingother}

\begin{description}
\item[Problem]~\\
GeoMason natively supports shape files and ARC/Info ASCII Grid files.
However, you have geospatial data that is in neither one of those
formats you would like to use.

\item[Solution]~\\
(To be written)
\begin{Verbatim}[frame=lines,framesep=5mm,numbers=left,commandchars=+\[\]]
(To be written)
\end{Verbatim}

\item[Discussion ]
\end{description}






\section{Writing Geospatial Data}
\label{sec:writingdata}

This section covers recipes involving saving geospatial data from
GeoMason constructs to files.

\subsection{Writing a Shape File}
\label{sub:writingshapefile}

\begin{description}
\item[Problem]~\\
You want to save a GeoMason vector field to a Shape file.

\item[Solution]~\\
Use \code{ShapeFileExporter.write()} to save the vector field to a Shape file.
\begin{verbatim}
ShapeFileExporter.write("foo", vectorField);
\end{verbatim}
\item[Discussion]~\\
Obviously it doesn't make sense to write a shape file for shape files
you've already read in because the data presumably hasn't changed.  This recipe is, instead, useful for scenarios
where you have a vector layer of data that you've created wholly
within your simulation and wish to save so that you can load it into a
proper GIS.  

\begin{figure}[ht]
  \centering
  \includegraphics[width=0.75\textwidth]{CampusWorld2.png}
  \caption{Snapshot of ``Campus World'' demo .}
  \label{fig:campusworld}
\end{figure}

Consider the GeoMason ``Campus World'' demo that has
agents moving along walkways.\footnote{This demo is included with
  GeoMason, and can be found as \file{sim.app.geo.campusworld}}  The buildings, walkways, and roads were
loaded from shape files, but the agents were created stochastically
from within the simulation and stored in their own \class{GeomVectorField}.  When the simulation ends a shape file
describing these agents is written out to a shape file that can be
loaded along with the original shape files for analysis.  These agents
have the following three attributes: their age, movement rate, and whether they are
student or faculty. Fig. \ref{fig:campusworld} depicts these shape files
after they were loaded into a GIS with the faculty agents rendered as blue triangles, students as
red dots, and their relative sizes scaled to their respective movement rates.

Shape files are not single files but are instead comprised of a few mandatory files with
the extensions \file{.shp}, \file{.dbf}, and \file{.idx}.  The first
argument to \method{ShapeFileExporter.write()} specifies the file name
prefix used to generate these files from the given
\class{GeomVectorField}.

There is an optional shape file that contains the coordinate reference
system and uses the file name extension \file{.prj}.  This function
does not write this file.\footnote{However, a future incarnation of
  GeoMason may do so.}  However, if the \class{GeomVectorField} was
itself sourced from a shape file, then its corresponding \file{.prj}
file can be copied over using the new file extension used in the call
to \method{write()}.  Alternatively, if the \class{GeomVectorField} was
\emph{not} read from a shape file, and so does not have a corresponding
\file{.prj} file, you may still be in luck if you loaded other vector
layers from shape files.  If that's the case, you can likely
arbitrarily use one of their \file{.prj} files.

If you want to automatically save a snapshot of a
\class{GeomVectorField} after each simulation run, you can place this
call to \method{write()} within \method{finish()} inside your
\class{SimState} subclass.
\end{description}



\subsection{Writing an ARC/Info ASCII Grid File}
\label{sub:writinggridfile}

\begin{description}
\item[Problem]~\\
You want to write GeoMason grid data to an ARC/Info ASCII Grid file.

\item[Solution]~\\
Create a \code{Writer} for the grid file and use \code{ArcInfoASCGridExporter.write()} to write the \code{GeomGridField}.
\begin{Verbatim}[frame=lines,framesep=5mm,numbers=left,commandchars=+\[\]]
try {
   BufferedWriter writer = new BufferedWriter( new FileWriter("foo.asc") );
   ArcInfoASCGridExporter.write(gridField, writer);
   writer.close();
} catch (IOException ex)  {
   /* handle exception */
}
\end{Verbatim}

\item[Discussion] ~\\
This recipe is useful for saving grid data from a simulation run such
that you can later import it into a GIS for analysis.

Unlike calls to \method{ArcInfoASCGridImporter.read()}, you do not
have to specify a data type.  Instead, the call to
\method{ArcInfoASCGridExporter.write()} automatically handles that
detail for you.

As in recipe \ref{sub:writingshapefile} you can place this snippet into
\method{finish()} to automatically write a grid file when the
simulation ends.
\end{description}




\chapter{Using Geospatial Data}
\label{ch:using}

In this chapter we discuss interacting with geospatial data using
GeoMason.  These kinds of interactions are mostly queries such as
asking in what political boundaries an agent is located or determining
nearby entities.

%\begin{figure}[h]\vspace{-33em}\hspace{30em}\includegraphics[width=4in]{ants.pdf}\vspace{2em}\end{figure}

\section{Determining What Political Entity an Agent is In}
\label{sec:politicalentity}

\begin{description}
\item[Problem]~\\
You have a simulation with polygons for boundaries that delineate political
entities such as countries,
counties, or voting districts.  In that simulation you also have
agents that move across these kinds of boundaries and you would like to know the
political entity in which that agent is located.

\item[Solution]~\\
Problem Solution
\begin{Verbatim}[frame=lines,framesep=5mm,numbers=left,commandchars=+\[\]]
	GeomGridField gridField = new GeomGridField();
	
	InputStream inputStream = new FileInputStream("foo.asc");
	ArcInfoASCGridImporter.read(inputStream, GridDataType.INTEGER, gridField);
\end{Verbatim}

\item[Discussion ]
\end{description}


\section{Locating Nearby Geospatial Objects}
\label{sec:nearbyobjects}

\begin{description}
\item[Problem]~\\
You want to find all the objects within a certain distance from a
specific thing.

\item[Solution]~\\
Problem Solution
\begin{Verbatim}[frame=lines,framesep=5mm,numbers=left,commandchars=+\[\]]
	GeomGridField gridField = new GeomGridField();
	
	InputStream inputStream = new FileInputStream("foo.asc");
	ArcInfoASCGridImporter.read(inputStream, GridDataType.INTEGER, gridField);
\end{Verbatim}

\item[Discussion ]
\end{description}


\section{Finding Adjacent Geospatial Objects}
\label{sec:findingadjacent}

\begin{description}
\item[Problem]~\\
You want to find adjacent geospatial objects. For example, for a given
country, you want to get a list of neighboring countries that share a
common border.

\item[Solution]~\\
Problem Solution
\begin{Verbatim}[frame=lines,framesep=5mm,numbers=left,commandchars=+\[\]]
	GeomGridField gridField = new GeomGridField();
	
	InputStream inputStream = new FileInputStream("foo.asc");
	ArcInfoASCGridImporter.read(inputStream, GridDataType.INTEGER, gridField);
\end{Verbatim}

\item[Discussion ]
\end{description}


\section{Moving Agents Along Paths}
\label{sec:movingalongpaths}

\begin{description}
\item[Problem]~\\
You have a path, such as a road or trail, along which you want to move
an agent.

\item[Solution]~\\
Problem Solution
\begin{Verbatim}[frame=lines,framesep=5mm,numbers=left,commandchars=+\[\]]
	GeomGridField gridField = new GeomGridField();
	
	InputStream inputStream = new FileInputStream("foo.asc");
	ArcInfoASCGridImporter.read(inputStream, GridDataType.INTEGER, gridField);
\end{Verbatim}

\item[Discussion ]
\end{description}



\section{Computing Line Intersections}
\label{sec:lineintersection}

\begin{description}
\item[Problem]~\\
You want to find all the intersections for a set of lines.  For
example, you may want to locate all road junctions.

\item[Solution]~\\
Problem Solution
\begin{Verbatim}[frame=lines,framesep=5mm,numbers=left,commandchars=+\[\]]
	GeomGridField gridField = new GeomGridField();
	
	InputStream inputStream = new FileInputStream("foo.asc");
	ArcInfoASCGridImporter.read(inputStream, GridDataType.INTEGER, gridField);
\end{Verbatim}

\item[Discussion ]
\end{description}



\section{How to Calculate the Shortest Path on a Network}
\label{sec:shortestpaths}

\begin{description}
\item[Problem]~\\
You want to find the shortest path between two points in a network
such as set of roads.

\item[Solution]~\\
Problem Solution
\begin{Verbatim}[frame=lines,framesep=5mm,numbers=left,commandchars=+\[\]]
	GeomGridField gridField = new GeomGridField();
	
	InputStream inputStream = new FileInputStream("foo.asc");
	ArcInfoASCGridImporter.read(inputStream, GridDataType.INTEGER, gridField);
\end{Verbatim}

\item[Discussion ]
\end{description}




\section{Calculating Agent Information from Underlying Grid Data}
\label{sec:gridtoagentdata}

\begin{description}
\item[Problem]~\\
You wish to to compute something based on values found in a grid an
agent occupies.  For example, a grid may represent a hectare and an
agent may want to consume certain amount of vegetation and water found there.

\item[Solution]~\\
Problem Solution
\begin{Verbatim}[frame=lines,framesep=5mm,numbers=left,commandchars=+\[\]]
	GeomGridField gridField = new GeomGridField();
	
	InputStream inputStream = new FileInputStream("foo.asc");
	ArcInfoASCGridImporter.read(inputStream, GridDataType.INTEGER, gridField);
\end{Verbatim}

\item[Discussion ]
\end{description}



\section{Having an Agent Follow a Gradient}
\label{sec:followinggradients}

\begin{description}
\item[Problem]~\\
You have slope information that you would like an agent to follow.

\item[Solution]~\\
Problem Solution
\begin{Verbatim}[frame=lines,framesep=5mm,numbers=left,commandchars=+\[\]]
	GeomGridField gridField = new GeomGridField();
	
	InputStream inputStream = new FileInputStream("foo.asc");
	ArcInfoASCGridImporter.read(inputStream, GridDataType.INTEGER, gridField);
\end{Verbatim}

\item[Discussion ]
\end{description}





%%
%%
%%

\chapter{Displaying Geospatial Data}
\label{ch:displaying}

This chapter gives recipes for displaying GeoMason fields in MASON.

%\begin{figure}[h]\vspace{-26em}\hspace{25.3em}\includegraphics[width=4in]{hexabugs.pdf}\vspace{2em}\end{figure}

%\noindent A {\bf grid} is MASON's name for objects arrange in a 2-dimensional or 3-dimensional array or equivalent.  MASON supports a wide range of grid environments as fields (representations of space).  This includes most any combination of the following:

\section{Displaying a \code{GeomVectorField}}
\label{sec:displayingGeomVectorField}

\begin{description}
\item[Problem]~\\
You want to show the contents of a \code{GeomVectorField} in a
MASON display.

\item[Solution]~\\
Create a \code{GeomVectorFieldPortrayal} in the MASON \code{GUIState}
subclass, associate it with its corresponding \code{GeomVectorField},
set up an appropriate MASON or GeoMason field portrayal, and attach it
to a MASON \code{Display2D} object.

\begin{Verbatim}[frame=lines,framesep=5mm,numbers=left,commandchars=+\[\]]
public class MyMasonGUI extends GUIState
{
    private Display2D display;
    private JFrame displayFrame;

    // ... other variable declarations

    +color[red]private GeomVectorFieldPortrayal myPortrayal = new GeomVectorFieldPortrayal();

    @Override
    public void init(Controller controller)
    {
        super.init(controller);

        display = new Display2D(ARBITRARY_WIDTH, ARBITRARY_HEIGHT, this);

        +color[red]display.attach(myPortrayal, "My Vector Layer");

        displayFrame = display.createFrame();
        controller.registerFrame(displayFrame);
        displayFrame.setVisible(true);
    }

    @Override
    public void start()
    {
        super.start();
        setupPortrayals();
    }

    private void setupPortrayals()
    {
        MyState world = (MyState)state;

        +color[red]myPortrayal.setField(world.vectorLayer);
        +color[red]myPortrayal.setPortrayalForAll(new GeomPortrayal(Color.CYAN, true));+label[ex:trueforfill]

        display.reset();
        display.setBackdrop(Color.WHITE);

        display.repaint();
    }

    // ... other code
}
\end{Verbatim}

\item[Discussion ]~\\

Line \ref{ex:trueforfill} creates a portrayal that draws all the lines
in cyan; the optional second parameter, which is set to \code{true}, indicates
that polygons should be filled.
\end{description}





\section{Displaying a \code{GeomGridField}}
\label{sec:displayinggridfield}

\begin{description}
\item[Problem]~\\
You want to show the contents of a \code{GeomGridField} in a MASON display

\item[Solution]~\\
Problem Solution
\begin{Verbatim}[frame=lines,framesep=5mm,numbers=left,commandchars=+\[\]]
foo
\end{Verbatim}

\item[Discussion ]
\end{description}



\section{Displaying Boundary Lines Over a Grid Field}
\label{sec:boundaryovergrid}

\begin{description}
\item[Problem]~\\
You want to overlay political boundaries on grid data.

\item[Solution]~\\
Problem Solution

\begin{Verbatim}[frame=lines,framesep=5mm,numbers=left,commandchars=+\[\]]
foo
\end{Verbatim}

\item[Discussion ]
\end{description}



\section{Displaying a Dynamic Choropleth Map}
\label{sec:chroplethmap}

\begin{description}
\item[Problem]~\\
You want to create a dynamic choropleth map with colors changing based
on agent behavior.

\item[Solution]~\\
Problem Solution
\begin{Verbatim}[frame=lines,framesep=5mm,numbers=left,commandchars=+\[\]]
	GeomGridField gridField = new GeomGridField();
	
	InputStream inputStream = new FileInputStream("foo.asc");
	ArcInfoASCGridImporter.read(inputStream, GridDataType.INTEGER, gridField);
\end{Verbatim}

\item[Discussion ]
\end{description}



\section{Displaying Raster Overlays}
\label{sec:displayingrasteroverlays}

\begin{description}
\item[Problem]~\\
You have raster overlays you wish to render over vector data.

\item[Solution]~\\
Problem Solution
\begin{Verbatim}[frame=lines,framesep=5mm,numbers=left,commandchars=+\[\]]
	GeomGridField gridField = new GeomGridField();
	
	InputStream inputStream = new FileInputStream("foo.asc");
	ArcInfoASCGridImporter.read(inputStream, GridDataType.INTEGER, gridField);
\end{Verbatim}

\item[Discussion ]
\end{description}






%%
%%
%%

\chapter{Common Problems}
\label{ch:commonprobs}


\section{Display All One Color}
\label{sec:OneColorDisplay}

\begin{description}
\item[Problem]~\\
Rendering a \code{GeomField} just shows one solid color.

\item[Solution]~\\
Problem Solution
\begin{Verbatim}[frame=lines,framesep=5mm,numbers=left,commandchars=+\[\]]
foo
\end{Verbatim}

\item[Discussion ]
\end{description}


\section{Layers Do Not Align}
\label{sec:NonAlignedLayers}

\begin{description}
\item[Problem]~\\
The data between layers does not match.

\item[Solution]~\\
Problem Solution
\begin{Verbatim}[frame=lines,framesep=5mm,numbers=left,commandchars=+\[\]]
foo
\end{Verbatim}

\item[Discussion ]
\end{description}


%%
%%
%%
\chapter{Acknowledgements}
\label{ch:ack}

%TODO add grant number and flesh out.
Thanks the MURI project for their support via NSF grant \#.  For Andrew Crooks for
valuable feedback.  For Sean Luke, et al, for MASON.

%%
%%
%%

\cleardoublepage
\footnotesize
\addcontentsline{toc}{chapter}{Index}
\printindex

\end{document}















